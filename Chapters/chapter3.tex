%!TEX root = ../template.tex
%%%%%%%%%%%%%%%%%%%%%%%%%%%%%%%%%%%%%%%%%%%%%%%%%%%%%%%%%%%%%%%%%%%%
%% chapter3.tex
%% NOVA thesis document file
%%
%% Chapter with a short latex tutorial and examples
%%%%%%%%%%%%%%%%%%%%%%%%%%%%%%%%%%%%%%%%%%%%%%%%%%%%%%%%%%%%%%%%%%%%

\typeout{NT FILE chapter3.tex}%

\makeatletter
\newcommand{\ntifpkgloaded}{%
  \@ifpackageloaded%
}
\makeatother


\chapter{Planning}
\label{cha:planning}

This Chapter aims at exemplifying how to do common stuff with \LaTeX. We also show some stuff which is not that common! ;)

Please, use these examples as a starting point, but you should always consider using the \emph{Big Oracle} (aka, \href{http://www.google.com}{Google}, your best friend) to search for additional information or al-ternative ways for achieving similar results.


\section{Our project} % (fold)
\label{sec:our_project}

% section our_project (end)


\section{Delegation Protocols} % (fold)
\label{sec:delegation_protocols}

Citing something online~\cite{wiki:shuntingyard,flex,bison}.

% section delegation_protocols (end)


\section{Evaluation Planning} % (fold)
\label{sec:evaluating_planning}

% section evaluating_planning (end)


\section{Scheduling} % (fold)
\label{sec:scheduling}

% section scheduling (end)


% \subsection{Inserting Figures Wrapped with text} % (fold)
% \label{ssec:inserting_images_wrapped_with_text}
%
% You should only use this feature is \emph{really} necessary. This means, you have a very small image, that will look lonely just with text above and below.
%
% In this case, you must use the \verb!wrapfigure! package.  To use \verb!wrapfig!, you must first add this to the preamble:
%
% \begin{wrapfigure}{l}{2.5cm}
%   \centering
%     \includegraphics[width=2cm]{snowman-vectorial}
%   \caption{A snow-man}
% \end{wrapfigure}
%
% \noindent\verb!\usepackage{wrapfig}!\\
% This then gives you access to:\\
% \verb!\begin{wrapfigure}[lineheight]{alignment}{width}!\\
% Alignment can normally be either ``l'' for left, or ``r'' for right. Lowercase ``l'' or ``r'' forces the figure to start precisely where specified (and may cause it to run over page breaks), while capital ``L'' or ``R'' allows the figure to float. If you defined your document as twosided, the alignment can also be ``i'' for inside or ``o'' for outside, as well as ``I'' or ``O''. The width is obviously the width of the figure. The example above was introduced with:
% \lstset{language=TeX, morekeywords={\begin,\includegraphics,\caption}, caption=Wrapfig Example, label=lst:latex_example}
% \begin{lstlisting}
%   \begin{wrapfigure}{l}{2.5cm}
%     \centering
%       \includegraphics[width=2cm]{snowman-vectorial}
%     \caption{A snow-man}
%   \end{wrapfigure}
% \end{lstlisting}

% subsection inserting_images_wrapped_with_text (end)

% section floats_figures_and_captions (end)



% \begin{algorithm}
% $i\gets 10$\;
% \eIf{$i\geq 5$}
% {
%     $i\gets i-1$\;
% }{
%     \If{$i\leq 3$}
%     {
%         $i\gets i+2$\;
%     }
% }
% \caption{This is an algorithm.}
% \end{algorithm}

% section test_for_listings (end)

% \printbibliography[heading=subbibliography, segment=\therefsegment, title={\bibname\ for chapter~\thechapter}]

