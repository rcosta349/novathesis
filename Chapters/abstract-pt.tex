%!TEX root = ../template.tex
%%%%%%%%%%%%%%%%%%%%%%%%%%%%%%%%%%%%%%%%%%%%%%%%%%%%%%%%%%%%%%%%%%%%
%% abstract-pt.tex
%% NOVA thesis document file
%%
%% Abstract in Portuguese
%%%%%%%%%%%%%%%%%%%%%%%%%%%%%%%%%%%%%%%%%%%%%%%%%%%%%%%%%%%%%%%%%%%%

\typeout{NT FILE abstract-pt.tex}%

Nos últimos anos, o aumento de aplicações baseadas em dados impulsionou o desenvolvimento de estruturas de aprendizagem descentralizadas, como a Aprendizagem Federada (FL) e a Aprendizagem Dividida (SL), concebidas para melhorar a privacidade dos dados e distribuir a carga computacional entre vários dispositivos . No entanto, os atuais sistemas de FL ainda dependem fortemente de agregadores centralizados, o que impõe limitações de privacidade e eficiência devido à orquestração centralizada. 

Esta tese propõe uma plataforma totalmente descentralizada que combina as metodologias FL e SL para ultrapassar estes desafios. 
A nossa abordagem projeta um sistema autogerido para aprendizagem automática descentralizada e colaborativa, minimizando a intervenção humana através de protocolos de delegação dinâmicos e mecanismos adaptativos de atribuição de funções. Os participantes podem treinar modelos de aprendizagem automática de forma colaborativa num ambiente seguro, com os dispositivos a lidarem autonomamente com alterações na rede, como falhas de dispositivos e alterações de conectividade.

A nossa investigação contribui para o campo da aprendizagem descentralizada, fornecendo uma alternativa robusta aos sistemas centralizados tradicionais, permitindo a colaboração e a adaptabilidade.

\keywords{
  plataforma totalmente descentralizada \and
  FL \and
  SL \and
  colaboração \and
  adaptabilidade
}
% to add an extra black line
